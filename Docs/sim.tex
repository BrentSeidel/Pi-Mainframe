\documentclass[10pt, openany]{book}

\usepackage{fancyhdr}
\usepackage{imakeidx}

\usepackage{amsmath}
\usepackage{amsfonts}

\usepackage{geometry}
\geometry{letterpaper}

\usepackage{fancyvrb}
\usepackage{fancybox}

\usepackage{url}
\usepackage{multicol}
%
% Rules to allow import of graphics files in EPS format
%
\usepackage{graphicx}
\DeclareGraphicsExtensions{.eps}
\DeclareGraphicsRule{.eps}{eps}{.eps}{}
%
% Macro definitions
%
%\newcommand{\operation}[1]{\textbf{\texttt{#1}}}
%\newcommand{\package}[1]{\texttt{#1}}
%\newcommand{\function}[1]{\texttt{#1}}
%\newcommand{\constant}[1]{\emph{\texttt{#1}}}
%\newcommand{\keyword}[1]{\texttt{#1}}
%\newcommand{\datatype}[1]{\texttt{#1}}
\newcommand{\switch}[2]{``#1''/``#2''}
\newcommand{\position}[1]{``#1''}
%
% Front Matter
%
\title{Raspberry Pi Based Mainframe Simulator}
\author{Brent Seidel \\ Phoenix, AZ}
\date{ \today }
%========================================================
%%% BEGIN DOCUMENT
\begin{document}
%
% Produce the front matter
%
\frontmatter
\maketitle
\begin{center}
This document is \copyright 2021 Brent Seidel.  All rights reserved.

\paragraph{}Note that this is a draft version and not the final version for publication.
\end{center}
\tableofcontents

\mainmatter
%----------------------------------------------------------
\chapter{Introduction}
A rather simplistic view is that this provides a way to get a Raspberry Pi to blink some LEDs and read some switches.  However, this hardware could be modified for a number of other applications that require lots of discrete I/O, such as burglar alarms or sprinkler controls

%----------------------------------------------------------
\chapter{Hardware}

\section{Electronics}
\subsection{Processor}
I had a Raspberry Pi 3 board laying around so I figured that I would use it.  Interestingly, this is a more powerful computer than many of the processors it might be used to simulate.

\subsection{Discrete I/O}
Each of these boards in built on a 1/2 size perma-proto board from Adafruit.  It has one MCP23017 16 bit I/O expander chip with I2C interface in a socket.  Three pins are provided to configure the address, so the system can have up to 8 of these chips or 128 discrete I/O pins.  There is space on the board for 8 current limiting resistors for LED drive.

\section{3D Printed Parts}
\subsection{Racks}
Each rack is a bbs\_rack2(10, 6, 4).  Two of these can be bolted together side-by-side and fit in a 19 inch rack.  This is the approximate size of the classic computer equipment.  Multiple racks can be bolted together vertically as needed to provide panel space.  Each rack has slots for two vertically oriented trays.

\subsection{Discrete I/O Tray}
Each of these trays has one or two MCP23017 16 bit I/O expander chips with I2C interface.  These trays would typically be mounted in the slot next to the panel.  The tray has 4 DB-9 connections and can interface with up to  16 LEDs and 16 switches.  One additional DB-9 connection is provided to connect the the power and I2C bus from the processor.

\subsection{Address/Data Panels}
The address/data panels provide 8 switches and LEDs numbered consecutively from right to left.  The starting number is selectable in the model.  Each panel has two cables (one for LEDs and one for switches) that attach to connectors on the discrete I/O boards.  Enough of these need to be provided to account for all the address and data lines in the simulated processor.  A system would typically have two or four of these panels for 16 or 32 bit systems.  Note that there are many cases where the number of address lines does not match the number of data lines.  In these cases panels would need to be provided for whichever number is greater.

\subsection{Control Panel}
At this point, without an actual CPU simulation, this is mostly a way to organize what needs to be in a relatively generic control panel.  The following functions are needed:
\begin{itemize}
  \item Read from a memory location.
  \item Write to a memory location.
  \item Stop/pause processor execution.
  \item Start processor execution at a specified address.
  \item Continue processor execution.
  \item Reset the processor.
\end{itemize}

\subsubsection{To Specify an Address}
Note that if the word size is greater than 8 bits, some of the lower order address bits may be ignored.
\begin{itemize}
  \item Toggle the desired address using the address/data switches.
  \item Move the \switch{Addr}{Data} switch to the \position{Addr} position.
  \item Move the ``Dep'' (deposit) switch to the \position{Dep} position.
  \item Once the LED for the ``Dep'' switch illuminates, move the switch to the unlabeled position.
\end{itemize}

\subsubsection{To Read Memory}
\begin{itemize}
  \item Specify the starting address as described above.
  \item Move the \switch{Addr}{Data} switch to the \position{Data} position.
  \item Move the ``Exam'' switch to the \position{Exam} position.
  \item The contents of memory at the specified address will be displayed in the Address/Data LEDs.
  \item The \switch{Addr}{Data}  switch can be used to toggle between display of the address and the data.
  \item Move the ``Exam'' switch to the unlabeled position.  This increments the internal address pointer so that by toggling this switch a block of memory can be modified.
\end{itemize}

\subsubsection{To Write Memory}
\begin{itemize}
  \item Specify the starting address as described above.
  \item Move the \switch{Addr}{Data}  switch to the \position{Data} position.
  \item Toggle the desired data using the address/data switches.
  \item Move the ``Dep'' switch to the \position{Dep} position.
  \item The contents of memory at the specified address will be updated and displayed in the Address/Data LEDs.
  \item The \switch{Addr}{Data} switch can be used to toggle between display of the address and the data.
  \item Move the ``Dep'' switch to the unlabeled position.  This increments the internal address pointer so that by toggling this switch a block of memory can be examined.
\end{itemize}

\subsubsection{To Start Execution at a Specific Address}
\begin{itemize}
  \item Move the \switch{Start}{Stop} switch to the \position{Stop} position.
  \item Move the \switch{Addr}{Data} switch to the \position{Addr} position.
  \item Toggle the address using the address/data switches.
  \item Move the \switch{Start}{Stop} switch to the \position{Start} position.
  \item If the \switch{Run}{Pause} switch is in the \position{Pause} position, move it to the \position{Run} position.
\end{itemize}

\section{Other Hardware}

%----------------------------------------------------------
\chapter{Software}
\section{Overview}

\section{Simulation}

\section{Web Server}

\end{document}

